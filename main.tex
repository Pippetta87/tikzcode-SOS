\documentclass[oneside,12pt,fleqn]{memoir}%% Grafici con Tikz per eliosismologia

%% colors
\usepackage[usenames,dvipsnames]{xcolor}
\definecolor{antiquefuchsia}{rgb}{0.57, 0.36, 0.51}
\definecolor{violetw}{rgb}{0.93, 0.51, 0.93}
\definecolor{Veronica}{rgb}{0.63, 0.36, 0.94}
\definecolor{atomictangerine}{rgb}{1.0, 0.6, 0.4}
\definecolor{darkgray}{rgb}{0.66, 0.66, 0.66}
\definecolor{brightcerulean}{rgb}{0.11, 0.67, 0.84}
\definecolor{cadmiumorange}{rgb}{0.93, 0.53, 0.18}
\definecolor{ochre}{rgb}{0.8, 0.47, 0.13}
\definecolor{midnightblue}{rgb}{0.1, 0.1, 0.44}
\definecolor{grey}{rgb}{0.7, 0.75, 0.71}
\definecolor{bf}{RGB}{88, 86, 88}
\definecolor{bb}{RGB}{177, 177, 177}

\usepackage{makeidx}
%\usepackage[utf8]{inputenc}
\pagestyle{plain}
%%%%%%%%%%%%%%%%%%%%%%%%%%%%%%%%%%% importa pacchetti
\usepackage{usepkg}
\usepackage{fancyfoot}
%%%%%%%%%%%%%%%%%%%%%%%%%%%%%%%%%%%%%
%%%%%%%%%%%%%%%%%% titletoc, titlesec setting
\usepackage{titleT}
%%%%%%%%%%%%%%%%%% setlength
\usepackage{mylength}
\linespread{0.5}
%%%%%%%%%%%% Hyperref package
\usepackage{hyperref}
\hypersetup{
    colorlinks,
    citecolor=black,
    filecolor=black,
    linkcolor=black,
    urlcolor=black
}
%%%%%%%%%%%%%%%%%%%%%%%%
%%%%%%%%%%%%%%%%%Geometry package
\usepackage{mygeometry}
%%%%%%%%%%%%%%%%%%%%%%%%%%%%%%%%%%% Funzioni per questo file main
\usepackage{LocalF}
%%%%%%%%%%%%%%%%%%%%%%%%%%%%%%%%%%% Funzioni generali
\usepackage{functions}
\usepackage{mathOp}
%http://tex.stackexchange.com/questions/246/when-should-i-use-input-vs-include
\usepackage{sources}
%%%%%%%%%%%%%%%%%%%%%%%%%%%%%%%%%

\makeindex
\raggedbottom %http://tex.stackexchange.com/questions/102084/annoying-paragraph-spacing-issue-with-memoir

\author{Pippetta}
\title{Grafici con Tikz per eliosismologia}
\date{\today}

%\input{rawdata}

\begin{document}

\maketitle
\tableofcontents*

\part{TIKZ}

%\begingroup
%\eject \pdfpagewidth=3in \pdfpageheight=5in

\chapter{Math-Phys function}

\section{Spherical harmonics}

%\input{SHF.tex}

\chapter{Modello solare}

\section{Composizione chimica}

%\subfile{chemel}

\section{reazioni nucleari}

\subsection{Catena PP}

\clearpage
\begin{comment}
\begingroup
\setmuskip{\thinmuskip}{0mu}\setmuskip{\medmuskip}{0mu}
\tikzset{->-/.style={decoration={
  markings,
  mark=at position .5 with {\arrow{>}}},postaction={decorate}},
-->/.style={decoration={
  markings,
  mark=at position .8 with {\arrow{>}}},postaction={decorate}},
box/.style={%
%draw,
minimum width=25mm,%
    minimum height=6mm,%
    align=center}
}

\begin{tikzpicture}

\node[box] (pp) at (0,0) {$\Pproton{+}\Pproton{\to}\cel{H}{2}{}{}{+}\Pnue{+}\APelectron$};%%pp
\node[box,right=2cm of pp]  (pep) {$\Pproton{+}\Pproton{+}\Pelectron{\to}\cel{H}{2}{}{}+\Pnue$};%%pep
\coordinate[below=0.3cm of pp] (bpp);
\node[left] at (bpp) {$99.76\%$};
\coordinate[below=0.3cm of pep] (bpep);
\node[right] at (bpep) {$0.24\%$};

\coordinate[] (ttriton) at ($(bpp)!0.5!(bpep)$);
\draw[->-] (pp)--(bpp)--(ttriton);
\draw[->-] (pep)--(bpep)--(ttriton);
\node[box,below=0.3cm of ttriton] (triton) {$\Pproton+\cel{H}{2}{}{}\to\cel{He}{3}{}{}+\Pphoton$};%%triton
\coordinate[below=0.3cm of triton] (btriton);
\draw[-->] (ttriton)--(triton.north);
\draw[->-] (triton.south)--(btriton.north);
\coordinate[left=2.5cm of btriton] (tpp1);
\node[left] at (tpp1) {$83.3\%$};
\coordinate[right=2.0cm of btriton] (tberillium7);
\node[above] at (tberillium7) {$16.7\%$};
\coordinate[right=6.5cm of btriton] (thep);
\node[right] at (thep) {$\num{2e-5}\%$};

\draw[] (btriton)--(tpp1);
\draw[] (btriton)--(tberillium7);
\draw[] (tberillium7)--(thep);
\node[box,below=0.5cm of tpp1,label={[xshift=0.1cm, yshift=-1.5cm]PPI}]  (pp1) {$\cel{He}{3}{}{}+\cel{He}{3}{}{}\to\cel{He}{4}{}{}+2\Pproton$};%%pp1
\node[box,below=0.5cm of tberillium7]  (berillium7) {$\cel{He}{3}{}{}+\cel{He}{4}{}{}\to\cel{Be}{7}{}{}+\Pphoton$};%%berillium7
\node[box,below=0.5cm of thep,label={[xshift=-0.1cm, yshift=-1.5cm]HEP}]  (hep) {$\cel{He}{3}{}{}+\Pproton\to\cel{He}{4}{}{}+\APelectron+\Pnue$};%%hep

\draw[->-] (tpp1)--(pp1.north);
\draw[->-] (tberillium7)--(berillium7.north);
\draw[-->] (thep)--(hep.north);

\coordinate[below=0.3cm of berillium7] (bberillium7);
\coordinate[left=1.5cm of bberillium7] (tlithium7);
\node[left] at (tlithium7) {$99.88\%$};
\coordinate[right=2.0cm of bberillium7] (tboron8);
\node[right] at (tboron8) {$0.12\%$};

\node[box,below=0.5cm of tlithium7]  (li7) {$\cel{Be}{7}{}{}+\Pelectron\to\cel{Li}{7}{}{}+\Pnue$};%%Li7
\node[box,below=0.5cm of li7,label={[xshift=0.1cm, yshift=-1.5cm]PPII}] (pp2) {$\cel{Li}{7}{}{}+\Pproton\to2\cel{He}{4}{}{}$};%% PP2

\node[box,below=0.5cm of tboron8]  (b8) {$\cel{Be}{7}{}{}+\Pproton\to\cel{B}{8}{}{}+\Pphoton$};%%B8
\node[box,below=0.25cm of b8]  (be7) {$\cel{B}{8}{}{}\to\cel{Be}{8}{}{}^*+\APelectron+\Pnue$};%%Be8*
\node[box,below=0.25cm of be7,label={[xshift=0.1cm, yshift=-1.5cm]PPIII}]  (pp3) {$\cel{Be}{8}{}{}^*\to2\cel{He}{4}{}{}$};%%pp3

\draw[->-] (berillium7.south)--(bberillium7);
\draw[] (bberillium7)--(tlithium7);
\draw[] (bberillium7)--(tboron8);

\draw[->-] (tlithium7)--(li7.north);
\draw[->-] (li7.south)--(pp2.north);

\draw[->-] (tboron8.south)--(b8.north);
\draw[->-] (b8.south)--(be7.north);
\draw[->-] (be7.south)--(pp3.north);

\end{tikzpicture}
\endgroup

\clearpage

\end{comment}

\subsection{Bi-ciclo CN-NO}

\section{Modello solare bahcall 95}

\clearpage

%\subfile{bahcalldiffusion}
\subfile{BS05AGSOP}

\section{Correzioni al MSS}

\clearpage
%\subfile{relationship}

%\cite{flowchart}


\chapter{Solar modes}

\section{Cavit\'a risonanti}

%\subfile{plotmodes}


\chapter{inversion}

\section{Inversione non asintotica}

%\subfile{inversiondelta}

\clearpage
\addcontentsline{toc}{section}{Index}
\printindex

\end{document}