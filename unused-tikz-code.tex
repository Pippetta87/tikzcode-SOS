
\begin{comment}
\begin{workout}[Stelle pulsanti]
\begin{wrapfigure}[23]{r}{0.6\textwidth}
\includegraphics[width=0.65\textwidth,keepaspectratio]{HRpulsating} \label{fig:HRp}
\caption{Zone del diagramma di \hr{} in cui sono previsti comportamenti oscillatori. Da \cite{dal03notes}.}
\end{wrapfigure}
\'E possibile che parte dell'energia interna di una stella alimenti un comportamento oscillatorio attorno alla configurazione di equilibrio: le frequenze dei modi normali sono le osservabili sperimentali che contengono informazioni dirette sull'interno stellare, oltre al flusso di neutrini prodotti nelle reazioni nucleari.
\end{workout}
\end{comment}


\node[name=hydrogen, right= 8mm of opint.south west] {\tiny H};
\node[name=helium, right=4.5mm of hydrogen.west] {\tiny He};
\node[name=carbonium, right=4.5mm of helium.west] {\tiny C};
\node[name=nitrum, right=4.5mm of carbonium.west] {\tiny N};
\node[name=oxygen, right=4.5mm of nitrum.west] {\tiny O};
\node[name=neon, right=4.5mm of oxygen.west] {\tiny Ne};
\node[name=sodium, right=4.5mm of neon.west] {\tiny Na};
\node[name=magnesium, right=4.5mm of sodium.west] {\tiny Mg};
\node[name=alluminium, right=4.5mm of magnesium.west] {\tiny Al};
\node[name=silicium, right=4.5mm of alluminium.west] {\tiny Si};
\node[name=sulfur, right=4.5mm of silicium.west] {\tiny S};
\node[name=argon, right=4.5mm of sulfur.west] {\tiny Ar};
\node[name=calcium, right=4.5mm of argon.west] {\tiny Ca};
\node[name=cromum, right=4.5mm of calcium.west] {\tiny Cr};
\node[name=manganese, right=4.5mm of cromum.west] {\tiny Mn};
\node[name=ferrum, right=4.5mm of manganese.west] {\tiny Fe};
\node[name=nikel, right=4.5mm of ferrum.west] {\tiny Ni};

%% opacity picture

\begin{tikzpicture}[remember picture,overlay]

\node[anchor=north west] (opint) at ([shift={(0,2.5)}]current page.north west) {\includegraphics[width=0.55\textwidth,keepaspectratio]{opcontrib-int-g}};
\node[anchor=north east] (opout) at ([shift={(0,2.5-3/35)}]current page.north east) {\includegraphics[width=0.55\textwidth,keepaspectratio]{opcontrib-out-g}};
\node[draw,anchor=north,label={[label distance=2mm]-90:Scattering \Pphoton\Pelectron},minimum size=5mm] (sc) at ([shift={(0,-7)}]current page.north) {};
\node[draw,label={[label distance=2mm]-90:ff},fill=black,minimum size=5mm,above=10mm of sc] (ff) {};
\node[draw,label={[label distance=2mm]-90:bb},fill=bb,minimum size=5mm,above=10mm of ff] (bb) {};
\node[draw,label={[label distance=2mm]-90:bf},fill=bf,minimum size=5mm,above=10mm of bb] (bf) {};


\node[name=hydrogen, right= 8mm of opint.south west] {\tiny H};
\node[name=helium, right=4.5mm of hydrogen.west] {\tiny He};
\node[name=carbonium, right=4.5mm of helium.west] {\tiny C};
\node[name=nitrum, right=4.5mm of carbonium.west] {\tiny N};
\node[name=oxygen, right=4.5mm of nitrum.west] {\tiny O};
\node[name=neon, right=4.5mm of oxygen.west] {\tiny Ne};
\node[name=sodium, right=4.5mm of neon.west] {\tiny Na};
\node[name=magnesium, right=4.5mm of sodium.west] {\tiny Mg};
\node[name=alluminium, right=4.5mm of magnesium.west] {\tiny Al};
\node[name=silicium, right=4.5mm of alluminium.west] {\tiny Si};
\node[name=sulfur, right=4.5mm of silicium.west] {\tiny S};
\node[name=argon, right=4.5mm of sulfur.west] {\tiny Ar};
\node[name=calcium, right=4.5mm of argon.west] {\tiny Ca};
\node[name=cromum, right=4.5mm of calcium.west] {\tiny Cr};
\node[name=manganese, right=4.5mm of cromum.west] {\tiny Mn};
\node[name=ferrum, right=4.5mm of manganese.west] {\tiny Fe};
\node[name=nikel, right=4.5mm of ferrum.west] {\tiny Ni};

\node[anchor=north west] at ([shift={(0,10.5)}]current page.north west) {\parbox{\textwidth}{\captionof{figure}{Importanza dei varii contributi all'opacit\'a nell'interno solare. Da \cite{bla11opacity}.}\label{fig:opacitycontrib} }};
 
\end{tikzpicture}

\begin{comment}%%Tabella fattori astrofisici
\begin{table}[!h]
\pgfplotstabletypeset[
%every head row/.style={before row={\toprule & Reaction & $S(0)$ & $S'(0)$ & $S''(0)$ & $\delta S(E)$\\\midrule},
%every last row/.style={after row=\bottomrule},
% after row={\midrule}
%},
col sep=&
every last row/.style={after row=\bottomrule},
every first column/.style={column type/.add={|}{}},
every last column/.style={column type/.add={}{|}},
%columns/0/.style = {column type/.add={|}{}},
%columns/xi/.style = {column type/.add={|}{}},
%display columns/0/.style={column name={r}},
%display columns/1/.style={column name={S}},
%display columns/2/.style={column name={Sst}},
%display columns/3/.style={column name={Snd}},
%display columns/4/.style={column name={deltaSE}}

%create on use/authors/.style={create col/set list={
%},
%columns/authors/.style={string type},
%columns={r,S,Sst,Snd,deltaSE},
%/pgf/number format/precision=4
]{reactionastrofactor.txt} %%%

\captionof{table}{Fattore astrofisico. Da \cite{adelberger2011solar}.}
\end{table}
\end{comment}

%La composizione chimica \'e modificata dalle reazioni di fusione che per gli elementi principali, assumendo condizione di equilibrio secolare, riassumo
%%% \tag{\theequation a,b}
%\begin{subequations}\label{subeqn:fusionchange}
%\begin{align}
%&\dot{X}=\frac{m_p}{N_A}(-3r_{pp}+2r_{33}-r_{34}-4r_{p14})\\ 
%&\dot{Y}_3=\frac{m_{He3}}{N_A}(r_{pp}-2r_{33}-r_{34})\\
%&\dot{Y}=\frac{m_{He4}}{N_A}(r_{33}+r_{34}+r_{p14})
%\end{align}
%\end{subequations}
%con $r_{ik}$ rate di reazione per unit\'a di massa.

\begin{table}[!h]

\begin{tabular}{|ccccc|}
{Reaction} & {$S(0) (keVb)$} & {$S'(0)$} & {$S''1(0) (bikeV)$} & {Gamow peak uncertainty (\%)}\\
{$p(p,\APelectron\Pnue)d$} & {$(4.01 \pm 0.04)10^{-22}$} & {$(4.49 \pm 0.05)10^{-24}$} & {$  $} & {$\pm 0.7$}\\
$d(p,\Pphoton)\cel{He}{3}{}{}$ & ${2.14}10^{-4}\substack{+0.17 \\ -0.16}$ & $ $ & $ $ & $\pm 7.1$\\
$\cel{He}{3}{}{}(\cel{He}{3}{}{},2p)\cel{He}{4}{}{}$ & $(5.21 \pm 0.27)10^{-3}$ & $-4.9 \pm 3.2$ & $(2.2 \pm 1.7)10^{-2}$ & $\pm 4.3$\\
$\cel{He}{3}{}{}(\cel{He}{4}{}{},\Pphoton)\cel{Be}{7}{}{}$ & $0.56 \pm 0.03$ & $(-3.6 \pm 0.2)10^{-4}$ & $(0.151 \pm 0.008)10^{-6}$ & $\pm 5.1$\\
$\cel{He}{3}{}{}(p,\APelectron\Pnue)\cel{He}{4}{}{}$ & $(8.6 \pm 2.6)10^{-20}$ & $$ & $$ & $\pm 30$\\
$\cel{Be}{7}{}{}(\Pelectron,\Pnue)\cel{Li}{7}{}{}$ & $ $ & $ $ & $ $ & $\pm 2.0$\\
$p(p\Pelectron,\Pnue)d$ & $ $ & $ $ & $ $ & $\pm 1.0$\\
$\cel{Be}{7}{}{}(p,\Pphoton)\cel{B}{8}{}{}$ & $(2.08 \pm 0.16)10^{-2}$ & $(-3.1 \pm 0.3)10^{-5}$ & $(2.3 \pm 0.8)10^{-7}$ & $\pm 7.5$\\
%14N(p,7)150 XI.A 1.66 \pm 0.12 (-3.3 \pm 0.2) x 10-3 b (4.4 \pm 0.3) x 10-5 a \pm 7.2
\end{tabular}
\caption{Reaction rate \cite{adelberger2011solar}.}
\end{table}